\glsresetall{} 
\appendix

\chapter{Algorithms}

Through the development of the \gls{pops}, a number of algorithms have been
developed to perform various functions. These algorithms are not necessarily
ground breaking but their implementations are novel and worth discussing in
some form. To avoid detracting from the main body of the thesis they have been
placed here, in the appendices.


\section{Multiple Access Intersection Algorithm} \label{alg:mul-access-inter}

For some search scenarios, we may need to determine for what times do all
satellites have access to a target such as an \gls{aoi} or a ground station.
That is, if we have more than one satellite and each satellite has a list of
access times to a target, we must generate a new list of access times where
each access corresponds to a period where all satellites have access. An
example scenario is illustrated in Figure \ref{fig:access_intersect}.


\begin{figure}[h]
    \includegraphics[width=\textwidth]{Access Intersection Example.png} 
    \caption{Illustration of a Potential Access Intersection Scenario}
\label{fig:access_intersect}
\end{figure}

In this scenario, there are three satellites: Sat 0, Sat 1, and Sat 2. Each has
multiple access periods represented with grey boxes. Though time is continuous,
it has been discretized into integer timesteps for simplicity. Minutes have
been selected as the units for time but this is arbitrary. From these lists of
access periods, this algorithm must determine all of the points in time where
all satellites have an access period. For the example scenario, the outputted
results should be $[7,10]$ and $[28,36]$. Note that between $t = 16$ and $t=22$
there are access overlaps but since there are only overlaps between two
satellites, they should not be returned as intersection periods.

For a satellite, access times are stored as a list of timestamps. Every even
and odd indexed timestamp specifies when the satellite `enters' and `leaves' an
access respectively. Access lists $\ses{a}{n}$ can be described generally for satellite $n$ as,

\begin{equation} 
    \ses{a}{n} = \left[ a_{n,0}, b_{n,0}, a_{n,1}, b_{n,1}, \ldots a_{n,m}, b_{n,m} \right]
\end{equation}


where $a$ is the access enter timestamp, $b$ is the access leave timestamp, and
$m$ is the number of accesses. We also make the assumption that no accesses
overlap for a given satellite and target; that is,

\[
    a_{n,0} < b_{n,0} < a_{n,1} < b_{n,1} < \ldots < a_{n,m} < b_{n,m}
\]

One simple brute force solution to this problem would be to iterate over every
time step, $t$, and check if there exists an access region, $[a_n,b_n]$, for all
satetellites, $n$, such that $t$ is between $a$ and $b$. Put succinctly,


This solution is, of course, very wasteful as it scales with the number of
timesteps that are being considered, $\Theta(t)$. The number of timesteps maybe
on the order of 100s to 10,000s of timesteps. A simplification can be made
since we do not need to actually consider every timestep. Rather, we can
instead iterate over the access boundaries since they describe continuous
periods of time. By focusing on just the access boundaries, we may develop an
algorithm which scales with the number of accesses, $\Theta(m)$, which is much
smaller than the totalnumber of timesteps.

For all satellite access lists we are considering, let us combine them into two
$1\times nm$ arrays. The first `timestamp' array, $\se{b}$, contains a sorted
list of all of the boundary timestamps in ascending order. The second `index'
array, $\se{s}$, contains a list of satellite indeces in the same order as the
timestamp array. For example,

\begin{equation*}
    \begin{aligned} 
	\ses{a}{0} &= \left[ 2, 10, 18, 21  \right] \\
	\ses{a}{1} &= \left[ 4, 12, 16, 19  \right] \\
	\ses{a}{2} &= \left[ 7, 15, 20, 22  \right] \\
    \end{aligned}
    \quad \Rightarrow \quad
    \left \{ 
	\begin{aligned}
	    \se{b} &= [ 2 , 4 , 7 , 10 , 12 , 15 , 16 , 18 , 19 , 20 , 21 , 22  ] \\
	    \se{s} &= [ 0 , 1 , 2 , 0 , 1 , 2 , 1 , 0 , 1 , 2 , 0 , 2  ]
	\end{aligned}
    \right.
\end{equation*}

Note these are some of the values from Figure \ref{fig:access_intersect}.
Again, in the actual implementation of the algorithm we use actual timestamps,
but here we are using integers for demonstration purposes. The timestamp array
stores the timestamp of the access boundary for later reference and also gives
us the order of the satellite index array. Now with the index array we can do
something interesting. Remember that access boundaries are listed in order of
[enter, leave, enter, leave, etc.]. Looking at the first four elements in the
index array, $[0, 1, 2, 0]$, satellite 0 enters an access at the first element
and leaves the access at the fourth element. So for the second and third
element, satellite 0 still has access because it has not left yet. In essence,
the index array encodes in what order satellites enter and leave accesses. 

Now let us expand on the index array so we can perform logical operations to
find intersections. For this make use of logic arrays.  These are arrays which
contain only boolean values, True or False. With these arrays, we can also
perform logical operations on any axis. For example, if we have a 2 dimensional
logic array, we can produce a 1 dimensional array, that is the result of
AND'ing all of the elements in each column. These allow us to perform logical
operations very quickly for many elements. From the index array let us
construct an $n\times nm$ boolean array that describes our scenario, $\se{A}$.
The rows of matrix, $\se{A}$, corrspond to the indeces of each satellite.  For
example row 0 is satellite 0, row $m$ is satellite $m$, etc. The columns
correspond to elements in the index array, $\se{s}$.

Let us initialize $\se{A}$ to be all False represented as 0s. Then, starting at
the first column of $\se{A}$, let us NOT the element in the $\se{s}(0)$ row.
Then, for then next column, let us copy all of the values from the previous
column and again NOT the $\se{s}(1)$ element. This is then repeated for all
columns in $\se{A}$. There is one small catch, if we are transitioning a 1 to a
0 or a True to a False, this should be done on the following iteration. This
essentially means that we are treating accesses in in access boundaries as
inclusive. Even if the satellite is leaving an access, we say that it has
access until the timestep imediately after the boundary. As an example, let us
construct $\se{A}$ from $\se{s}$ for all of Figure \ref{fig:access_intersect},

\begin{equation*} 
    \se{s} = 
    \left[
    \begin{array}{cccccccccccccccccc}
	0 & 1 & 2 & 0 & 1 & 2 & 1 & 0 & 1 & 2 & 0 & 2 & 1 & 0 & 2 & 1 & 0 & 2 \\
    \end{array}
    \right]
\end{equation*}
yields,
\begin{equation*} 
    \se{A} = 
    \left[
	\begin{array}{cc;{2pt/2pt}cc;{2pt/2pt}cccccccccc;{2pt/2pt}cc;{2pt/2pt}cc}
	1 & 1 & 1 & 1 & 0 & 0 & 0 & 1 & 1 & 1 & 1 & 0 & 0 & 1 & 1 & 1 & 1 & 0 \\
	0 & 1 & 1 & 1 & 1 & 0 & 1 & 1 & 1 & 0 & 0 & 0 & 1 & 1 & 1 & 1 & 0 & 0 \\
	0 & 0 & 1 & 1 & 1 & 1 & 0 & 0 & 0 & 1 & 1 & 1 & 0 & 0 & 1 & 1 & 1 & 1 \\
    \end{array}
    \right]
\end{equation*}
Then, if we AND all of the rows in $\se{A}$ we get,
\begin{equation*} 
    \se{A}' = 
    \left[
    \begin{array}{cccccccccccccccccc}
	0 & 0 & 1 & 1 & 0 & 0 & 0 & 0 & 0 & 0 & 0 & 0 & 0 & 0 & 1 & 1 & 0 & 0 \\
    \end{array}
    \right]
\end{equation*}
Now it is clear to see that this matrix gives us the indeces where there is an
intersection between all satellites. If we take all values of $\se{b}$ where
$\se{A}'$ is True, we are left with,
\begin{equation*} 
    \se{b}' = 
    \left[
    \begin{array}{cccccccccccccccccc}
	7 & 10 & 28 & 32
    \end{array}
    \right]
\end{equation*}
Which is our expected result. This was just a walkthrough but the explicit
alogrithm definition is as follows,

\begin{algorithm}[h] 
    \caption{Access Intersection} 
    \label{alg:access-intersection}
    \begin{algorithmic}[1]
	%\Require{\se{z}is a $1\times N$ array } 
	\Function{AccessIntersection}{$\ses{a}{0}$,$\ses{a}{1}$, ... , $\ses{a}{n}$} 

	    \Let{$\se{s}$, $\se{b}$}{\Call{Combine}{$\ses{a}{0}$,$\ses{a}{1}$, ... , $\ses{a}{n}$}}  

	    \Let{$l$}{\Call{Length}{$\se{s}$}}

	    \Let{$\se{A}$}{\Call{Zeros}{$n$,$l$}}

	    %\Let{$\se{A}[\se{s}[0],0]$}{1}  \Comment{Set up the first column}
	    \Let{$temp$}{$\se{A}[:,0]$} \Comment{Temporary array to store column of $\se{A}$}

	    \Let{$i$}{0} \Comment{Boundary iterator}

	    \While{$i \neq m$}
		\Let{$s$}{$\se{s}[i]$} \Comment{Satellite index}
		
		\If{!$temp$($s$)}
		    \Let{$temp$($s$)}{!$temp$($s$)}
		    \Comment{Flip element then copy values over}
		    \Let{$\se{A}[:,s]$}{\Call{OR}{$\se{A}[:,s]$, $temp$}} 
		\Else
		    \Let{$\se{A}[:,s]$}{\Call{OR}{$\se{A}[:,s]$, $temp$}} 
		    \Comment{Copy values over then flip element}
		    \Let{$temp$($s$)}{!$temp$($s$)}
		\EndIf


		\Let{$i$}{$i+1$}
	    \EndWhile 

	    \Let{$\se{A}'$}{\Call{ColumnsAND}{$\se{A}$}}

	    \Let{$\se{b}'$}{$\se{b}[\se{A}']$}

	\State \Return $\se{b}'$
	\EndFunction
    \end{algorithmic}

\end{algorithm}

%%%%%%%%%%%%%%%%%%%%%%%%%%%%%%%%%%%%%%%%%%%%%%%%%%%%%%%%%%%%%%%%%%%%%%%%%%%%%% 

\section{Equator Crossing Algorithm} \label{sec:equator-crossing}

For a given ephemeris, it is useful to determine each `pass' of that orbit.
That is, for each ephemeris point, an index should be assigned to it which
indicates how many times the spacecraft has orbited the Earth. In this way, if
we have some time range and we wish to see the next `pass,' we would simply
take that time range's pass index and add 1.

There are many ways a pass may be defined. For example we could specify a
latitude and longitude range and whenever the spacecraft is in this range, that
could be considered a singular pass.  Generally, though, epehemeris data is not
given in latitude or longitude, rather it is given in a cartesian position in
some \gls{eci} or \gls{ecef} reference frame.  So for each position in the
ephemeris, the position vector will need to be converted to latitude and
longitude. 

This is a completely acceptable approach but we may also simplify the problem.
Instead of taking a latitude and longitude range, we could instead increment
the pass index when the spacecraft crosses the equator and goes from the
southern to the northern hemisphere. This would be when the spacecraft's
position goes from a negative to a positive latitude. This definition of a pass
has a few advantages. That being, we only need to do one check to determine a
pass boundary. It also has the benefit of indexing the entire ephemeris. Still
for this method, we need to convert from cartesian postiion to at least
latitude.

Let us make one further simplification by assuming that the x-y plane of the
ephemeris's coordinate system is very near to the Earth's equitorial plane.
This is not true for all cooordinate systems but it is true for the \gls{ecef}
ephemerides used by \gls{pops}. By making this assumption, we no longer need to
calculate the latitude of the spacecraft; rather, we can instead only look at
the spacecraft's position along the z-axis. This is useful because the
spacecraft's z-position will oscillate between some positive and negative
extrema, which are determined by the orbit's inclination and excentricity. 

There is a complicating factor that should be accounted for. In time, the
spacecraft's position is periodic, but when considering only the spacecraft's
z-position in an ephemeris, there is no guarantee that there is a constant
timestep between position values. Additional data-points may be injected for
periods where greater accuracy is desired and vice-versa. We may now articulate
the problem to be addressed.

% TODO: Add two figures, one showing a sinusoidal z-comp in time, and one
% showing one with additional data points added, making it oblong.

Given, an array of $z$-positions, $\se{z}$, generate
that each element of $\se{p}$ is the index of an element in $\se{z}$ after a
crossover occurs (ie. the positive value in the negative-to-positive
crossing).


To determine if elements, $n$ and $m$, of an array, $\se{z}$, form a crossover,
they must satisfy three simple conditions:

\begin{enumerate}
    \item $\se{z}[n] < 0$
    \item $\se{z}[m] > 0$ 
    \item $m = n+1$
\end{enumerate}


A brute-force approach to finding $\se{p}$ would be to loop through all of the
elements in $\se{z}$, and test them against the above conditions. This method
is inelegant and may be computationally intensive, though.

One approach to this problem is outlined in the following Algorithm. In
essence, it attempts to reduce the number of comparissons made to search for
crossovers. 

\begin{algorithm}[h] 
    \caption{Negative-Positive Crossing Search Algorithm} 
    \label{alg:crossover}
    \begin{algorithmic}[1] 
	\Require{\se{z}is a $1\times N$ array } 

	\Function{FindAllCrossovers}{$s, f, \se{z}$}
	    \Let{$p$}{\Call{FindNextCrossover}{$0, s, f, \se{z}$}} \Comment{Start at beginning of array}
	    \Let{$\se{p}$}{$\{ p \}$}

	    \While{$p \neq -1$}
		\Let{$p$}{\Call{FindNextCrossover}{$p, s, f, \se{z}$}}
		\Comment{Start at beginning of array} \State
		$\se{p}$.append($p$) \EndWhile 
		\State \Return \se{p}
	\EndFunction

	\State

	\Function{FindNextCrossover}{$i, s, f, \se{z}$} 
	\Let{$j$}{$i+s$}
	\If{$j > length(\se{z})$} 
	    \State $s = \mathrm{s \times f}$  
	\ElsIf{$j = length(\se{z})$}
	    \State \Return $-1$	  \Comment{Search has completed}
	\ElsIf{ $(\se{z}[i] < 0) \lor (\se{z}[j] > 0)$ }
	    \If{$s=1$}
	    \State \Return $i$ \Comment{Crossover index found}
	    \Else
		\State $s = ceil(s \times f)$
	    \EndIf
	\Else
	    \State $i=j$
	\EndIf
	\State \Return \Call{FindNextCrossover}{$i, s, f, \se{z}$}
	\EndFunction 
    \end{algorithmic} 
\end{algorithm}


%%%%%%%%%%%%%%%%%%%%%%%%%%%%%%%%%%%%%%%%%%%%%%%%%%%%%%%%%%%%%%%%%%%%%%%%%%%%%% 

\section{Single Access Intersection Algorithm} \label{alg:contains}

%%%%%%%%%%%%%%%%%%%%%%%%%%%%%%%%%%%%%%%%%%%%%%%%%%%%%%%%%%%%%%%%%%%%%%%%%%%%%% 

\section{Swath Boundary Ellipse Algorithm} \label{alg:ellipse}

To review, Swaths are the area covered by a satellite's \gls{fov} footprint or
\gls{for} access region over time. They are calculated from a satellite
ephemeris, where for each telemetry point in the ephemeris, two swath boundary
points are generated by drawing rays some angle from the Nadir and intersecting
them with a WGS84 Ellipsoid. Therefore, a swath is represented by two
polylines, OFF-1 and OFF-2. 


%%%%%%%%%%%%%%%%%%%%%%%%%%%%%%%%%%%%%%%%%%%%%%%%%%%%%%%%%%%%%%%%%%%%%%%%%%%%%% 

\section{Counter-Clockwise Reordering Algorithm} \label{alg:ccw}

%%%%%%%%%%%%%%%%%%%%%%%%%%%%%%%%%%%%%%%%%%%%%%%%%%%%%%%%%%%%%%%%%%%%%%%%%%%%%% 

\section{Convex Polygon Conversion} \label{alg:force-complex}

