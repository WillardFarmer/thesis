\glsresetall{} 
\chapter{Introduction}
%\subsection{Small Satellites and New Space}


\section{Thesis Overview}


\lettrine[lines=2, findent=0pt, nindent=5pt]{O}{} ver the past decade, there
has been a sharp rise in the development of small spacecraft. With 112
nanosatellites launched in 2012 and 1800 launched in 20221, it is clear that
the industry is growing. As technology improves, so too do the capabilities of
small spacecraft. For missions with relatively ‘simple’ operations, it is
economical to use competent engineers to handle orbit operations. But, as
missions become more complex, more of the operator’s time is spent on repeated
calculations using purpose-built scripts that offer little in terms of
reusability. For suitably complicated operations, the economy of human
computation is lost. In this scenario, the benefits of software automation
become appealing. Commercial-off-the-shelf tools do exist to address this
issue. A ready example would be Orbit Logic’s \gls{cpaw}2 or their Scheduler
extension for the Ansys \gls{stk}3. These solutions are well-developed and have
a proven mission heritage. They are also quite expensive and still require
adapting to a specific mission scenario.  \\ 


At the \gls{sfl}, the \gls{pops} is being developed to streamline operations
planning for remote sensing missions. \gls{pops} is a general offline software
meant to handle the deterministic aspects of mission planning.  Given an
\gls{aoi} and a method of remote sensing, \gls{pops} presents possible
observation opportunities to the operator and creates sets of commands to be
uploaded to a spacecraft in orbit.  Ideally, \gls{pops} should prevent an
operator from having to do any calculations themselves manually. Not only is
\gls{pops} meant to be a tool used by \gls{sfl} operators but one of the goals
of this project is to reduce the barrier to development as much as possible.
The easier the underlying code is to work with, the more likely it is that
future developers work on it and the tool grows. For this reason, most of the
underlying code for \gls{pops} has been written in Python and leverages several
open-source libraries.  Extending this principle, \gls{pops} uses a
browser-based \gls{gui} rather than as an executable desktop application. As a
browser-based application, \gls{pops} can leverage an extensive set of existing
libraries for building functionality.  For example, CesiumJS4 may be used for
graphical Earth visualizations.  \gls{pops}’s architecture is containerized for
easy deployment, using Docker5.  Each service handles separate aspects of the
software’s functionality.  In this way, they can be developed separately and
switched out as needed. With services, functionality may also be integrated
into other applications. For persistent storage of planning information, an SQL
database has been implemented as its own service.  To facilitate searching for
observation opportunities, a suite of software tools has been developed, known
as the \gls{atu}.  \gls{pops} also handles its orbital propagation by making use of
open-source Python implementations of the SGP4 algorithm. 



\subsection*{Thesis Outline} 


\section{Existing Solutions}\label{sec:exsoln}

This is not a problem that is unique to Gray Jay Pathfinder.
For any mission with sufficiently complicated operations, developing or purchasing a license for planning software is prudent.
There are a multitude of existing solutions, each with their own benefits and shortcomings.




\subsection{\acrshort{cpaw}}

The \gls{cpaw} is the one of the most sophisticated operations planning tools on the market. 
Developed by Orbit Logic, \gls{cpaw} is one of their products available on the open market. 
Others being: SpyMeSat Mobile App, Order Logic, STK Scheduler, UAV Planner, and On-board Autonomous Planning System.
\gls{cpaw} has been used by a number of missions for their imaging planning operations: Landsat 8/9 \cite{gokhale_mission_nodate}, Worldview-1/2, GeoEye-1, and RADARSAT-2 \cite{herz_eo_2014}.


\gls{cpaw} specializes specifically on image planning. 
Once set up by an operator, it can generate operations schedules and can generate commands to be sent to spacecraft in orbit.
It supports both automatic and manual operations.







