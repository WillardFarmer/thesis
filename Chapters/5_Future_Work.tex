\glsresetall{} 

\chapter{Discussion and Future Work}

\lettrine[lines=2, findent=0pt, nindent=5pt]{A}{}t the time of writing this
thesis, the \gls{pops} is still very much in development. The project began in
2021 and has undergone continued work since then. When designing something that
is completely unique and as ambitious as a mission planning tool, it is
difficult to guage the work required to meet its requirements. Before,
discussing future directions for the tool, it would be helpful to first go over
how \gls{pops} became what it is. 

\subsection{Research and the Development Process}

Early on in development, much effort was put into exploring possible tools,
languages, and frameworks to build off of. Intitially, before the creation of
the Propagator service, the intention was to rely on an open-source tool
developed by NASA called the \gls{gmat}. \gls{gmat} is well made and designed
to model, optimise, and estimate spacecraft trajectories. Originally, it was
thought that it could act as a replacement for \gls{stk}. It could be used to
take \glspl{tle} and generate ephemerides. \gls{gmat} comes with its own
propagators and the software can be run from the command line. Given a script
representation of a mission, ephemeris data could be generated in the form of a
report. Unfortunately not one among thewas SGP4 which made \glspl{tle} unusable
given this approach. \gls{gmat} can be expanded and has a very well set up
plugin interface with the rest of the software. Theoretically, an open-source
implementation of an SGP4 propagator could be implemented into the software and
it could have been used. Then the question arose, if we were adding in our own
propagator, what was \gsl{gmat} providing. The answer, of course, was that it
created a great deal of work with no benefit. Some issues were: \gls{gmat} was
developed for a windows environment, data needed to be transfered through a
\gls{tcp} connection, generating mission scripts was convoluted, and several
more. This is not to say that this time was wasted; rather, this was time spent
exploring a, what seemed to be, useful tool that ended up being inadequate for
\gls{sfl}'s purposes. It was, for all intents and purposes, research.

%https://gmat.atlassian.net/wiki/spaces/GW/overview
%https://core.ac.uk/download/pdf/80605179.pdf

Similar work was done for another NASA tool called, OpenSPIFe. Its purpose was
to handle scheduling and planning. Very similar \gls{gmat}, it provided a great
deal of functionality that could be updated and developed. Similarly, this tool
would have increased the amount of work for any developer working on the
project. For any tool that is used, developers must take responsibility for it
to at least understand how to use the tool but also make changes if bugs
appear. For both \gls{gmat} and OpenSPIFe, it was determined that the
development overhead for developing the tools from scratch would be far less
than having to maintain two tools which are fully fledged projects in their own
rights, developed by teams of professional software engineers.

Since the decision was made to move to develop the tool from scrath,
development became more productive than exploratory. The first service
developed was the Propagator service. Ephemeris data is fundemantal to every
aspect of \gls{pops} so it was the top priority. Here was a situation where an
open-source library was more helpful than it was burdensome. Implementations in
many languages (including Excel) of the SGP4 algorithm are available along with
its source document, \hl{`Revisiting Spacetrack Report #3'}. These
implementations, though, are only of the algorithm itself and not any
surrounding functionality such as coordinate frame transforms or generating an
entire ephemeris. All of the heavy lifting was already done but the surrounding
implementation was missing. This is where a separate open-source implementation
of SGP4 came in very helpful. It was based on the same source implementation
and it provided a library of helper functions that enabled all of the
surrounding functionality. This allowed efforts to be focused on validation and
on actually using the ephemeris data rather than generating it.

\hl{cite all this}

A quick note on licensing. It is of the utmost importance that a developer
conforms with the license associated with a library lest they become exposed to
litigation if caught. This boils down to whether a library requires that the
source code of the software that makes use of it must be disclosed to the
public. A `copyleft' license requires that derivative works must disclose their
source code. The \gls{gpl} license is a very common copyleft license.
Conversly, a `permissive' license makes no obligations to derivative works. The
\gls{MIT} license and Apache License are very common examples of permissive
licenses. With this in mind, much time was spent ensuring that all of the open
source tools used by \gls{pops} fell under the permissive category. 







%\section{Thermal Design}

