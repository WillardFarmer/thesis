\glsresetall{} 
\chapter{Introduction}\label{chap:intro}
%\subsection{Small Satellites and New Space}



\lettrine[lines=2, findent=0pt, nindent=5pt]{O}{} ver the past decade, there
has been a sharp rise in the development of small spacecraft. With 112
nanosatellites launched in 2012 and 1800 launched in 2021
\cite{nanosats_total_2023}, it is clear that the industry is growing. As
technology improves, so too do the capabilities of small spacecraft. For
missions with relatively ‘simple’ operations, it is economical to use competent
engineers to handle orbit operations. But, as missions become more complex,
more of the operator’s time is spent on repeated calculations using
purpose-built scripts that offer little in terms of reusability. For suitably
complicated operations, the economy of human computation is lost. In this
scenario, the benefits of software automation become appealing.
\gls{cots} tools do exist to address this issue. A ready example
would be Orbit Logic’s \gls{cpaw} \cite{orbit_logic_cpaw_2021} or their
Scheduler extension for the Ansys \gls{stk} \cite{ansys_stk_nodate}. These
solutions are well-developed and have a proven mission heritage. They are also
quite expensive and still require adapting to a specific mission scenario.   


At the \gls{sfl}, the \gls{pops} is being developed to streamline operations
planning for remote sensing missions. \gls{pops} is a general offline software
meant to handle the deterministic aspects of mission planning.  Given an
\gls{aoi} and a method of remote sensing, \gls{pops} presents possible
observation opportunities to the operator and creates sets of commands to be
uploaded to a spacecraft in orbit.  Ideally, \gls{pops} should prevent an
operator from having to do any manual calculations themselves. Not only is
\gls{pops} meant to be a tool used by \gls{sfl} operators but one of the goals
of this project is to reduce the barrier to development as much as possible.
The easier the underlying code is to work with, the more likely it is that
future developers work on it and the tool grows. For this reason, most of the
underlying code for \gls{pops} has been written in Python and leverages several
open-source libraries.  Extending this principle, \gls{pops} uses a
browser-based \gls{gui} rather than an executable desktop application. As a
browser-based application, \gls{pops} can leverage an extensive set of existing
libraries for building functionality.  For example, CesiumJS may be used for
graphical Earth visualizations.  \gls{pops}’s architecture is containerized for
easy deployment, using Docker.  Each service handles separate aspects of the
software’s functionality.  In this way, they can be developed separately and
switched out as needed. With services, functionality may also be integrated
into other applications. For persistent storage of planning information, an SQL
database has been implemented as its own service.  To facilitate searching for
observation opportunities, a suite of software tools has been developed, known
as \gls{atu}.  \gls{pops} also handles its orbital propagation by making
use of open-source Python implementations of the \gls{sgp4} algorithm. 


\section{Existing Solutions}\label{sec:exsoln}

Mission planning is not unique to \gls{sfl}. Leveraging software automation
should be done for any mission with sufficiently complicated operations.  As
such, existing mission planning solutions already exist. For these existing
solutions though, the microspace design approach should be kept in mind. Even
though there may exist software that finds the best possible solutions with the
best algorithms and interface, can the same results be achieved through a more
limited but still satisfactory solution? 


\subsection{Collection Planning and Analysis Workstation}

The \acrfull{cpaw} is the one of the more sophisticated operations planning
tools available on the market.  Developed by Orbit Logic, \gls{cpaw} has
mission heritage and has been used for planning imaging operations for: Landsat
8/9~\cite{gokhale_mission_2019}, Worldview-1/2, GeoEye-1, and
RADARSAT-2~\cite{herz_eo_2014}.  Typically, \gls{cpaw}'s use case is for medium
to large constellations of synthetic-aperture RADAR or electro-optical
satellites. It combines dynamic modeling of the spacecraft with optimized
schedule planning. These schedules can then be used to generate commands to be
sent to the spacecraft. Planning may either be done manually or automatically. 

This is a very appealing solution but it has some limitations. Firstly,
\gls{cpaw} is expensive and licensing the software would require yearly
payments per mission. License fees do not necessarily preclude the use of
\gls{cpaw}, rather it is the fact that not all of \gls{cpaw}'s capabilities are
necessary. Secondly, \gls{sfl} does not require a tool that fully automates
satellite operations, rather it needs a tool that automates repetitive
calculations.  Operators will still need to be actively setting plans for
satellites.  \gls{cpaw} also provides a suite of proprietary algorithms that
produce an optimized operations schedule.  While this would be nice to have, an
optimal schedule is equivalent to a satisfactory schedule if both meet their
operational requirements. Time and money spent towards approaching optimality
may be wasted. In addition, out of the box, \gls{cpaw} does not provide support
for all of the operations modes performed at \gls{sfl}. These modes would need
to be developed and integrated into the software, which adds development costs
on top of the licensing costs. A further limitation is that the use of
\gls{cpaw} is restricted to only the mission that it is licensed for and cannot
be extended to future missions without new licensing.


\subsection{STK Scheduler} 

STK Scheduler is an extension to the Ansys \acrlong{stk} developed by orbit
logic. It provides the scheduling utility of \gls{cpaw} but utilizes the tools
provided by \gls{stk}. That is, the scheduling algorithms build off of the
orbit dynamics from \gls{stk} to generate optimal schedules. Operators at
\gls{sfl} are already very experienced with \gls{stk}, so utilizing a plugin
that supports schedule creation is appealing.

Similar to \gls{cpaw}, \gls{stk} Scheduler also requires an annual license fee.
If mission planning only needs to be done for a single mission, this would be a
good cost-effective solution. But, for multiple missions, they must either use
the same license, which may become challenging logistically, or multiple
licenses would need to be procured.


\subsection{Mission Planning and Scheduling Software}

The \gls{mpss} was developed for the NEMO-AM mission by \gls{sfl} in
2013~\cite{mehradnia_design_2013}. NEMO-AM did not launch but the \gls{mpss}
was later used for DMSat-1 which launched in 2021. The \gls{mpss}'s
purpose was to facilitate operations planning for Earth observation missions at
\gls{sfl}.  It was designed with the NEMO-AM mission in mind and incorporated:
orbit visualization, attitude control algorithms, star tracker data, \gls{gps}
data, orbit propagation, and spacecraft command generation.  Having an already
built in-house solution would be an ideal foundation to build a new software
from. 

There are a number of drawbacks to using the \gls{mpss}, though. The first
issue is that the software was developed in 2013 and built to run on the
\textit{RedHat Enterprise Linux 5} operating system. The operating system was a
specific requirement from the customer and is not used or supported by
\gls{sfl}. At the very least using the tool would require updating all of the
libraries to newer versions. Some libraries may not even be supported 9 years
later. Building off the \gls{mpss} also locks development to decisions made
that were reasonable in 2013, but may note be valid for \gls{sfl}'s current
purposes.  For example, should the new software be a desktop application using
Qt for its \gls{gui}, or should the new software directly handle attitude
control sensor data? These are very mission specific and potentially limit the
software's extensibility. Lastly, the \gls{mpss} was designed with a single
operation mode in mind, ground target tracking. In this mode, the satellite
would point its imager at a single ground target while it is visible to the
satellite. This is not sufficient for \gls{sfl}'s purposes as the new software
solution should be extensible to different operations modes.    


\section{Objectives}

Now that some existing solutions as well as their drawbacks have been
discussed, it is worth laying out the objectives for creating an operations
planning tool.  Specifically, what features are required for \gls{pops}? 

Suppose a customer comes to \gls{sfl} and describes an area on Earth they
wish to be observed through remote sensing. An operator must then construct a
payload operations plan to acquire information about that area. Without a tool,
they must manually perform calculations themselves to find potential
opportunities and create commands to be sent to the satellites. The purpose of
\gls{pops} is to streamline this process by:

\begin{enumerate} 

    \item Taking the area of interest and showing the possible observation
	opportunities to the operator,

    \item Allowing the operator to add observations based on the displayed
	opportunities,

    \item Validating the observations in a schedule to prevent conflicts, and

    \item Converting the planned observations to a list of satellite commands.

\end{enumerate}

Ideally, \gls{pops} should remove the manual calculation from the operator's
workflow. It should be noted that \gls{pops} will not fully automate operations
and is not meant to be a real time software. Rather, it must plan the
\textit{deterministic} aspects of a mission as far into the future as
necessary.  The reason for this is that full automation is much more difficult
and the validation requirements are much more stringent. This scenario has
greater risks associated with it since it would be a shift to relying on
software to operate a satellite rather than a human operator. 

\newcolumntype{P}[1]{>{\centering\arraybackslash}p{#1}}

\begin{table}[h]
    \centering
    \caption{Summary of key Objective for \gls{pops}}
    \label{tab:objectives}
    \begin{tabular}{|P{0.11\textwidth}|p{0.8\textwidth}|}
\hline
    \textbf{Objective} & \textbf{Description} \\ \hline
    \textbf{O1} &  Plan and schedule payload observations          \\ \hline
    \textbf{O2} &  Allow for semi-automated planning, such as enabling an operator to direct that a specific type of observation be repeated on a schedule           \\ \hline
    \textbf{O3} & Generate lists of commands to be uploaded to the spacecraft          \\ \hline
    \textbf{O4} &  Display the schedules of the spacecraft in a timeline format   \\ \hline
    \textbf{O5} & Have a graphical Earth display that shows: satellite orbits, areas of interest, observation constraints, observable areas, and observation opportunities. \\ \hline
    \textbf{O6} & Generalized enough to be expandable to different missions with minimal effort  \\ \hline
\end{tabular}
\end{table}

The objectives for \gls{pops} have been more clearly listed in
Table~\ref{tab:objectives}. This is not an exhaustive list of all of the
desired features for \gls{pops} but they do touch on the key objectives. These
are not only limited to the desired inputs and outputs of the tool but also how
the tool must display information. \gls{pops} must have the ability to
effectively display information to the user. Specifically, it must have a
graphical Earth display as well as timeline capabilities.


\section{Thesis Outline} 

The purpose of this thesis is to summarize the development of the
\acrlong{pops}. As well, its purpose is to highlight some of the challenges
that go along with creating a mission planning software. In
Chapter~\ref{chap:ops}, some operations concepts are discussed such as
terminology, orbit propagation, and the use of \glspl{ttc}. An example scenario is
also introduced to demonstrate how \gls{pops} may be used without disclosing
\gls{sfl}'s customers' operations strategies. In
Chapter~\ref{chap:architecture}, the software architecture of \gls{pops} is
discussed. This is the largest chapter as it goes into the different services
provided by \gls{pops}. In Chapter~\ref{chap:workflow}, this thesis walks through how
\gls{pops} may be used for the example scenario introduced in
Chapter~\ref{chap:ops}. Then in Chapter~\ref{chap:discussion} there is a
discussion and the future work for \gls{pops} is laid out. Finally, the thesis
is concluded in Chapter~\ref{chap:conclusion}.  After the conclusion, some
algorithms of note are discussed in the appendix.

%\newpage
%
%%On my first day of undergraduate studies, the Dean of Engineering at my
%%university gave a speech welcoming us. His one piece of advice was to 
%
%There are too many people to name that have either directly or indirectly
%contributed to this thesis. 
%
%First, I would like to extend my deepest thanks to my supervisor, Dr. Robert
%Zee. It is through both his direct and indirect leadership that has pushed me
%to become a better engineer. His advice and guidance has helped me expand my
%perspective on what constitutes a good engineering solution to a problem. I am
%fortunate to be part the environment that he has created at the Space Flight
%Laboratory which serves as an incubator for next generation of space engineers.
%Thank you for seeing my potential and giving me the opportunity to push my
%boundaries and thrive.
%
%I would next like to thank my mission manager Nathan Cole. I always felt as
%though Nathan made myself and the other students working under him a priority.
%He would always have time to discuss any problem and ensured that my work was
%not only productive and useful to the mission but also thoroughly enriching.
%With Nathan, I would also like to thank the other members of the POPS team:
%Matt, Lukasz, and Erica. This thesis was made possible through your
%collaboration, patience, and advice.  
%
%To my friends, you have made this master's program not only possible but two of
%the best years of my life. To my cohort: Diana, Andreas, Lukasz, Djordje, Kim,
%Connor, Hooman, Arjun, Eric, and Jacob, I have enjoyed every moment learning
%and sharing this experience with you. I know you will all excel wherever you
%go. To Jordy, since we met on the first day of second year of undergrad, you
%have been a constant companion. Thank you for your patience, support, and
%friendship. To Mia, thank you for showing me what is possible through talent,
%grit, and determination. Your example has pushed me work harder and be better.
%To my friends from home: Momo, Saymon, Gurjeet, Julie, Ethan, and Daylen, thank
%you for helping me keep my sense of community. Your friendship has served as my
%bedrock. 
%
%Oftentimes omitted in academics is physical exercise. Coming to the Space
%Flight Laboratory, has helped me transition from a post-Covid sedentary
%lifestyle to a more healthy and active one. For this, I would like to thank the
%members of the Swole Patrol: Nathan, Rami, Suraj, and Matt. Through mentorship
%and example, they have taught me how to make very heavy things feel less heavy.
%
%Lastly and the most importantly of all, I would like to thank Mum and Dad. I
%find it difficult to put into words the appreciation I have for your
%unconditional support. It feels as though you're always there, always know
%what's best and always know what to do. Team Farmer.



