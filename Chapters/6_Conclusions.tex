\glsresetall{} 

\chapter{Conclusion}\label{chap:conclusion}

\lettrine[lines=2, findent=0pt, nindent=5pt]{T}{} his thesis has discussed
the purpose, design, and implementation of the \gls{pops} developed by the
\gls{sfl}. \gls{pops} is being developed to address the need for mission
planning capabilities for the operations at \gls{sfl}. It is not intended to
fully automate an operator's workflow but rather its purpose is to streamline
some of their day to day activities. The main design direction for this tool
has been to keep it easy use and easy to develop. Many tools have been
developed at \gls{sfl} that have fallen into obscurity. Even if the tool is
very efficient, this is less valuable than a tool that can be expanded into the
future. 

Existing software solutions to mission planning that are currently available
have been introduced. Their strengths as well as their weeknesses have been
discussed. After this some terminology was introduced as well as some
operations related concepts. Since it is not the desire of the author to expose
the operations strategies of \gls{sfl} customers, an example scenario was
defined, EG-SAT, to illustrate the usefullness of \gls{pops}.

After this, the architecture of \gls{pops} was discussed. Starting with the
general architecture of the software. Then going into the specifics of each
service, those being the: Propagator, Database, Access Time Utilities, NSP
service, and Mission Model. Each has its own purpose and the largest is the
Mission Model. It handles the front-end user interface, communication with the
database, opportunity filtering, data handling, and event scheduling.

Just discussing the architecture does not fully explain the purpose of
\gls{pops}. To help aid in this process, a workflow was discussed for the
EG-SAT mission. Starting with setting up \gls{pops} for a mission, adding
satellites, and ground stations. After this, a plan was set up for a 3-day
period. There, opportunities were searched for given EG-SAT's two operations
modes, Coarse Imaging and Tip-and-Cue Imaging. From these opportunities,
observations were created and added to the schedule. The schedule was then
validated and displayed.  

A great deal of work lies ahead for \gls{pops} but this is only a testament to
the amount of work that has already been accomplished. Hopefully, this tool
will at least provide some benefit to operators at \gls{sfl} for many years to
come.










