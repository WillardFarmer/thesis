\glsresetall{} 
\chapter{Operations Concepts}

\lettrine[lines=2, findent=0pt, nindent=5pt]{B}{} efore discussing the
\gls{pops} it is necessary to go over basic terminology. Without doing so, it
may become very easy for descriptions to become unclear or imprecise. Some of
these terms have been illustrated in Figure \ref{fig:terminology}

\section{General Definitions}

\begin{description} 

    \item[Ephemeris] is a time series of orbital state vectors in a given
	coordinate system. These state vectors have a position component,
	$\sev{r}$, a velocity component, $\sev{v}$, and an epoch for each
	vector, $e$. See section \ref{sgp4_section} for a more detailed
	discussion on ephemerides.

    \item[Area of Interest] An \gls{aoi} is a region on the Earth that is of
	some particular importance to an operator. It is the region in which
	one or more spacecraft should observe through some means. \glspl{aoi}
	can be specified as a point location, an area target, or a latitude
	range. Some examples of \glspl{aoi} are illustrated in \ref{fig:AOIs}.

    \item[Field of View] The \gls{fov} is the extent to which a sensor or
	instrument may observe the outside world at a given time. The size and
	shape of an \gls{fov} varies based on the design of the sensor or
	instrument. \glspl{fov} can theoretically describe any volume of space
	but, for the purposes of this thesis, it may be assumed that they are
	conical. Specifically, the \gls{fov} is defined by a single half-angle,
	$\theta$. Suppose an instrument is pointing along some vector,
	$\sev{u}$. Let us define another vector, $\sev{v}$ such that the angle
	between it and $\sev{u}$ is $\theta$. The \gls{fov} is the volume
	described by rotating $\sev{u}$ around $\sev{v}$.


    \item[Field of Regard] The \gls{for} is similar to the \gls{fov} but
	instead of being the area a sensor can observe in a single time
	instant, the \gls{for} is the volume of space a sensor can possibly
	observe by changing its orientation. Typically, it is constrained by
	some physical or operational constraint. It is not possible to observe
	the entire \gls{for}; Rather, only a subset of the \gls{for} can be
	observed. This subset is the instrument's \gls{fov}.  For example, let
	us consider an optical sensor fixed to a spacecraft orbiting the earth.
	The position of the sensor at particular time instant is given by the
	spacecraft's orbit and cannot be changed unless a propulsive manoeuvre
	is performed.  Of course, the position of sensor can be changed
	slightly by changing the attitude of the spacecraft, since the
	instrument is most likely not located at the spacecraft's centre of
	mass. But, this can be ignored since the distance the instrument can
	translate is negligible compared to its orbit.  Conversely, The
	orientation of the instrument can be changed, and this has a meaningful
	effect on the \gls{for}. If no constraints are put on the attitude of
	the spacecraft, the \gls{for} is everywhere, since the instrument can
	be pointed in any direction. This is not always true though so let us
	say the spacecraft can only point an angle, $\alpha$, off nadir.  The
	\gls{for} would then be the cone described by the half-angle $\alpha +
	\theta$, where $\theta$ is again the half-angle of the conical sensor.
	Figure \ref{fig:fovfor} for an illustration of this example.  The
	larger blue cone is the \gls{for}. The red cone is the sensor's actual
	\gls{fov}. Its boresight is offset from the blue cone's.


\begin{figure} 
    \centering
    \begin{minipage}[c]{0.45\textwidth}
	\centering
	\includegraphics[width=\textwidth]{terminology.png} 
	\caption{General Illustration of Terminology}
	\label{fig:terminology} 
    \end{minipage}
    \hfill
    \begin{minipage}[c]{0.45\textwidth}
	\centering
	\includegraphics[width=\textwidth]{AOIs.png} 
	\caption{Different Types of Areas of Interest}
	\label{fig:AOIs} 
    \end{minipage} 
\end{figure}

    \item[Footprint] The footprint is the area on Earth that can be observed by
	an instrument's \gls{fov} or \gls{for} at a given time. It can be found
	by intersecting the \gls{fov} or \gls{for} with the surface of the
	Earth.  These intersection points form a boundary and the enclosed area
	within this boundary is the footprint. For clarity, it should be
	assumed that footprint refers to an \gls{for}'s footprint, unless
	otherwise specified.

    \item[Swath] If a sensor is moving over time, its footprint will move with
	it. A swath is the union of all footprints over a time range.  It is
	the region on Earth that can be possibly observed at some point by the
	sensor.

    \item[Horizon Swath] A horizon swath is a special case where the entire
	Earth is within the sensor's \gls{for}. This may be true for certain
	\gls{rf} payloads. In this case, the sensor can only `see' up until the
	horizon. That is, the `horizon' footprint is all of the points on Earth
	whose tangent line intersects with the sensor. Again, as the horizon
	footprint moves, this forms the horizon swath.

\end{description}



\section{Time Tag Commands}

At the Space Flight Laboratory (SFL), satellites are commanded through the use
of the Nanosatellite Protocol (NSP). NSP commands are a custom standard
developed by SFL to facilitate ground and intra-satellite communication. They
are designed to minimize the effects of low-bandwidth radio communication links
that are prone to error. These commands handle all aspects of nominal
operation, from turning individual units on or off, to specifying attitude
modes or initiating data transfers. Once a command is sent, it is executed
immediately upon being received. This raises the concern of how can a
spacecraft be commanded when it does not have a direct communication link with
a ground station. To address this, there exist Time Tag Commands (TTCs). TTCs
are NSP commands that have been prepended with a timestamp and a group ID. When
the spacecraft’s system clock reaches the time specified in the timestamp, that
command is executed. The group ID allows operators to group TTCs such that they
may be considered as a collection rather than as separate commands, allowing
for reference or removal as a single unit. TTCs are prepared in advance by an
operator and uploaded in bulk to the spacecraft. This allows operators to
control the spacecraft when direct communication cannot be established.

At the \gls{sfl}, satellites are commanded through the use of the \gls{nsp}.
\gls{nsp} commands are a custom standard developed by \gls{sfl} to facilitate
ground and intra-satellite communication. They are designed to minimize the
effects of low-bandwidth radio communication links that are prone to error.
These commands handle all aspects of nominal operation, from turning individual
components on or off, to specifying attitude modes or initiating data
transfers. Once a command is sent, it is executed immediately upon being
received. This, of course, poses an obvious problem, how should a spacecraft be
commanded when it does not have a direct communication link with a ground
station. To address this, there exist \glspl{ttc}. These are \gls{nsp} commands
that have been prepended with a timestamp. When the spacecraft's internal clock
reaches the time specified in the timestamp, that command is exectuted. These
\glspl{ttc} are prepared in advance by an operator and uploaded in bulk to the
spacecraft. This allows operators to control the spacecraft when direct
communication cannot be established.



% This should be moved somewhere else
\section{Equator Crossing Algorithm}

For a given ephemeris, it is useful to determine each ``pass'' of that orbit.
That is, for each ephemeris point, an index should be assigned to it which
indicates how many times the spacecraft has orbited the Earth. In this way, if
we have some time range and we wish to see the next `pass,' we would simply
take that time range's pass index and add 1.

\subsection{Problem}

There are many ways a pass may be defined. For example we could specify a
latitude and longitude range and whenever the spacecraft is in this range, that
oculd be considered a singular pass.  Generally, though, epehemeris data is not
given in latitude or longitude, rather it is given in a cartesian position in
some \gls{eci} or \gls{ecef} reference frame.  So for each position in the
ephemeris, the position vector will need to be converted to latitude and
longitude. 

This is a completely acceptable approach but we may also simplify the problem.
Instead of taking a latitude and longitude range, we could instead increment
the pass index when the spacecraft crosses the equator and goes from the
southern to the northern hemisphere. This would be when the spacecraft's
position goes from a negative to a positive latitude. This definition of a pass
has a few advantages. That being, we only need to do one check to determine a
pass boundary. It also has the benefit of indexing the entire ephemeris. Still
for this method, we need to convert from cartesian postiion to at least
latitude.

Let us make one further simplification by assuming that the x-y plane of the
ephemeris's coordinate system is very near to the Earth's equitorial plane.
This is not true for all cooordinate systems but it is true for the ephemerides
used by \gls{pops}. By making this assumption, we no longer need to calculate
the latitude of the spacecraft; rather, we can instead only look at the
spacecraft's position along the z-axis. This is useful because the spacecraft's
$z$-position will oscillate between some positive and negative extrema, which
are determined by the orbit's inclination and excentricity. 

There is a complicating factor that should be accounted for, though. In time,
the spacecraft's position is periodic, but when considering only the
spacecraft's $z$-position in an ephemeris, there is no guarantee that there is
a contant timestep between position values. Additional data-points may be
injected for periods where greater accuracy is desired and vice-versa.  We may
now articulate the problem to be addressed.

% TODO: Add two figures, one showing a sinusoidal z-comp in time, and one
% showing one with additional data points added, making it oblong.

\begin{center}
\fbox{\begin{minipage}{35em}
    \textbf{Problem:} Given, an array of $z$-positions, $\se{z}$, generate
    that each element of $\se{p}$ is the index of an element in $\se{z}$ after a
    crossover occurs (ie.  the positive value in the negative-to-positive
    crossing).
\end{minipage}}
\end{center}

\subsection{Solution}

To determine if elements, $n$ and $m$, of an array, $\se{z}$, form a crossover,
they must satisfy three simple conditions:

\begin{center}
\begin{varwidth}{\textwidth}
\begin{enumerate}
    \item $\se{z}[n] < 0$
    \item $\se{z}[m] > 0$ 
    \item $m = n+1$
\end{enumerate}
\end{varwidth}
\end{center}

\begin{enumerate}
    \centering
    \item  \parbox{0.5\linewidth}{$\se{z}[n] < 0$ }
    \item  $\se{z}[m] > 0$ 
    \item $m = n+1$
\end{enumerate}

A brute-force approach to finding $\se{p}$ would be to loop through all of the
elements in $\se{z}$, and test them against the above conditions. This method
is inelegant and may be computationally intensive, though.

One approach to this problem is outlined in Algorithm \ref{alg:crossover}. In
essence, it attempts to reduce the number of comparissons made to search for
crossovers. 


\begin{algorithm}[h] 
    \caption{Negative-Positive Crossing Search Algorithm} 
    \label{alg:crossover}
    \begin{algorithmic}[1] 
	\Require{\se{z}is a $1\times N$ array } 

	\Function{FindAllCrossovers}{$s, f, \se{z}$}
	    \Let{$p$}{\Call{FindNextCrossover}{$0, s, f, \se{z}$}} \Comment{Start at beginning of array}
	    \Let{$\se{p}$}{$\{ p \}$}

	    \While{$p \neq -1$}
		\Let{$p$}{\Call{FindNextCrossover}{$p, s, f, \se{z}$}}
		\Comment{Start at beginning of array} \State
		$\se{p}$.append($p$) \EndWhile 
		\State \Return \se{p}
	\EndFunction

	\State

	\Function{FindNextCrossover}{$i, s, f, \se{z}$} 
	\Let{$j$}{$i+s$}
	\If{$j > length(\se{z})$} 
	    \State $s = \mathrm{s \times f}$  
	\ElsIf{$j = length(\se{z})$}
	    \State \Return $-1$	  \Comment{Search has completed}
	\ElsIf{ $(\se{z}[i] < 0) \lor (\se{z}[j] > 0)$ }
	    \If{$s=1$}
	    \State \Return $i$ \Comment{Crossover index found}
	    \Else
		\State $s = ceil(s \times f)$
	    \EndIf
	\Else
	    \State $i=j$
	\EndIf
	\State \Return \Call{FindNextCrossover}{$i, s, f, \se{z}$}
	\EndFunction 
    \end{algorithmic} 
\end{algorithm}






















