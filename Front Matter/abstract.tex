\doublespacing 

As small spacecraft become more capable, so does the complexity of their
operations. Searching for potential concurrent observation or access
opportunities and ensuring they are compatible with one another may become
tedious and repetitive for operators to manually compute. Currently available
commercial-off-the-shelf tools that automate this process are capable but
expensive. To address this problem, a new payload operations planning tool has
been developed by the Space Flight Laboratory to handle the deterministic
aspects of mission planning, such as: detecting observation opportunities,
validating observations in a schedule, and generating lists of commands to be
sent to satellites. This lightweight tool is generalizable to any
Earth-observing mission configuration and can support complicated observation
geometries. Open-source libraries were used to reduce the overhead for
development as they decrease the amount of code that must be newly created and
maintained. Functionality has been compartmentalised through a containerized
service-based architecture. In this way, new functionality can be added or
replaced as needed. To enhance usability, a user may interact with the tool
through a browser-based user interface. This paper outlines the features of the
Payload Operations Planning Software, as well as details about its architecture
and development.

\doublespacing
