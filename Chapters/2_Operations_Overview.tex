\glsresetall{} 
\chapter{Operations Concepts}

\lettrine[lines=2, findent=0pt, nindent=5pt]{B}{} efore discussing the
\gls{pops} it is necessary to go over basic terminology.  Without doing so, it
may become very easy for descriptions to become unclear or inprecise.

\section{Definitions}

\begin{description} 

    \item[Area of Interest] An \gls{aoi} is a general concept for some region
	on the Earth that is of some particular importance to a user. It is the
	region which one or more spacecraft should be used to gain more
	information about the area. \glspl{aoi} can be specified as a point
	target, a polygon, or a latitiude range. See section \ref{aoi_def} for
	an exhaustive definition in this context. 

    \item[Field of View] The \gls{fov} is the extent a sensor or instrument may
	observe the outside world at a given time. The size and shape of an
	\gls{fov} varies based on the design of the sensor or instrument in
	question. Unless otherwise specified, for the purposes of this thesis,
	it may be assumed that \glspl{fov} are conical. Specifically, the
	\gls{fov} is defined by a single half-angle, $\theta$. Suppose an
	instrument is pointing along some vector, $\vec{u}$. Let us define
	another vector, $\vec{v}$ such that the angle between it and $\vec{u}$
	is $\theta$. The \gls{fov} is the volume described by rotating
	$\vec{u}$ around $\vec{v}$.

    \item[Field of Regard] The \gls{for} is similar to the \gls{fov}. Instead
	of the area an instrument can observe in a singular time instant, the
	\gls{for} is the area an instrument can possibly observe by changing
	its prientation given an external constraint. It is not possible to
	observe the entire \gls{for}. Rather, only a subset of the \gls{for}
	can be observed. This subset is the instrument's \gls{fov}. For
	example, let us consider an optical sensor fixed to a spacecraft
	orbiting the earth. The position of the sensor at particular time
	instant is given by the spacecraft's orbit and cannot be changed unless
	the spacecraft's orbit is changed. Of course, the position of sensor
	can be changed slightly by changing the attiutde of the spacecraft,
	since the instrument is most likely not located at the spacecraft's
	centre of mass. But, this can be ignored since the distance the
	instrument can translate is negligeable compared to its orbit. The
	orientation of the instrument can be changed, though, and this has a
	meaningful effect on the \gls{for}. If no constraints are put on the
	attitude of the spacecraft, the \gls{for} is everywhere, since the
	instrument can be pointed in any direction. Let us say the spacecraft
	can only point an angle, $\alpha$, off nadir. The \gls{for} would then
	be the cone described by the half-angle $\alpha + \theta$, where theta
	is again the half-angle of the conical sensor. \textbf{(MAKE A NEW
	'EXAMPLE SECTION')}
 

    \item[Footprint] The footprint is the area on Earth that can be observed by
	an instrument's \gls{fov} or \gls{for} at a given time. It can be found
	by intersecting the \gls{fov} or \gls{for} with the surface of the
	Earth.  These intersection points form a boundary and the enclosed area
	within this boundary is the footprint. For clarity, it should be
	assumed that footprint refers to an \gls{for}'s footprint, unless
	otherwise specified.

    \item[Swath] A swath is the union of all footprints over a time range for a
	particular sensor.


    \item[Horizon Swath] A horizon swath is a special case where the entire
	Earth is within the \gls{for}. 




	
\end{description}



